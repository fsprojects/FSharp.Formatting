\documentclass{article}

\usepackage{xcolor}
\usepackage{graphicx}
\usepackage{fancyvrb}
\usepackage{listings}
\usepackage[T1]{fontenc}
\usepackage{hyperref}
\usepackage{amsmath}

\definecolor{officegreen}{rgb}{0, 0.5, 0}
\definecolor{navy}{rgb}{0, 0, 0.5}
\definecolor{linecolor}{rgb}{0.5, 0.6875, 0.6875}
\definecolor{outputcolor}{rgb}{0.375, 0.375, 0.375}

\newcommand{\id}[1]{\textcolor{black}{#1}}
\newcommand{\com}[1]{\textcolor{officegreen}{#1}}
\newcommand{\inact}[1]{\textcolor{gray}{#1}}
\newcommand{\kwd}[1]{\textcolor{navy}{#1}}
\newcommand{\num}[1]{\textcolor{officegreen}{#1}}
\newcommand{\ops}[1]{\textcolor{purple}{#1}}
\newcommand{\prep}[1]{\textcolor{purple}{#1}}
\newcommand{\str}[1]{\textcolor{olive}{#1}}
\newcommand{\lines}[1]{\textcolor{linecolor}{#1}}
\newcommand{\fsi}[1]{\textcolor{outputcolor}{#1}}
\newcommand{\omi}[1]{\textcolor{gray}{#1}}

% Overriding color and style of line numbers
\renewcommand{\theFancyVerbLine}{
\lines{\small \arabic{FancyVerbLine}:}}

\lstset{%
  backgroundcolor=\color{gray!15},
  basicstyle=\ttfamily,
  breaklines=true,
  columns=fullflexible
}

\title{{page-title}}
\date{}

\begin{document}

\maketitle

\section*{Upgrading to fsdocs}



Here are the typical steps to upgrade a repo based on \texttt{ProjectScaffold} to use \texttt{fsdocs}
\begin{enumerate}
\item 

Run
\begin{Verbatim}[commandchars=\\\{\}]
\id{dotnet} \kwd{new} \id{tool}
\id{dotnet} \id{tool} \id{install} \id{FSharp}{.}\id{Formatting}{.}\id{CommandTool}

\end{Verbatim}


\item 

Delete all of \texttt{docs{\textbackslash}tools} particularly \texttt{docs{\textbackslash}tool{\textbackslash}generate.fsx}.  Keep a copy of any templates for reference as you'll have to copy some bits across to the new template.

\item 

Put your docs directory so it refelcts the final shape of the site. For example move the content of \texttt{docs{\textbackslash}input{\textbackslash}*} and \texttt{docs{\textbackslash}files{\textbackslash}*} directly to \texttt{docs{\textbackslash}*}

\item 

Create \texttt{docs{\textbackslash}\_template.html}, starting with \href{https://github.com/fsprojects/FSharp.Formatting/blob/master/misc/templates/\_template.html}{this file} and
copying across any snippets from old templates.

\item 

Run
\begin{Verbatim}[commandchars=\\\{\}]
\id{dotnet} \id{fsdocs} \id{watch}

\end{Verbatim}



and edit and test your docs.

\item 

If using FAKE adjust \texttt{build.fsx} e.g.
\begin{Verbatim}[commandchars=\\\{\}]
\id{Target}{.}\id{create} \str{"GenerateDocs"} {(}\kwd{fun} \id{\_} \kwd{->}
   \id{Shell}{.}\id{cleanDir} \str{".fsdocs"}
   \id{DotNet}{.}\id{exec} \id{id} \str{"fsdocs"} \str{"build --clean"} \ops{|>} \id{ignore}
{)}

\end{Verbatim}


\item 

Consider creating \texttt{docs{\textbackslash}\_template.fsx} and \texttt{docs{\textbackslash}\_template.ipynb} to enable co-generation of F\# scripts and F\# notebooks.


If you add support for notebooks and scripts, consider adding mybinder links to each of your literate executable content pages. \href{https://github.com/fsprojects/FSharp.Formatting/blob/master/docs/literate.fsx\#L19}{For example like this}.


Also add load sections to make sure your notebooks and scripts contain the right content to load packages out of repo.  \href{https://github.com/fsprojects/FSharp.Formatting/blob/master/docs/literate.fsx\#L1}{For example like this}

\end{enumerate}



Sample commands:
\begin{Verbatim}[commandchars=\\\{\}]
\id{dotnet} \id{tool} \id{install} \id{FSharp}{.}\id{Formatting}{.}\id{CommandTool} \ops{--}\id{local}
\id{git} \id{add} \id{dotnet}\ops{-}\id{tools}{.}\id{json}   
\id{git} \id{rm} \ops{-}\id{fr} \id{docs}
\id{git} \id{mv} \id{docs}
\id{git} \id{mv} \id{docs}

{<}\id{manually} \id{download} \kwd{and} \id{fixup} \id{the} \id{\_template}{.}\id{html}{>}

\id{dotnet} \id{fsdocs} \id{watch}

\id{touch} \id{docs}
\id{touch} \id{docs}
\id{git} \id{add} \id{docs}
\id{git} \id{add} \id{docs}

\end{Verbatim}



Here is an example PR: \href{https://github.com/fsprojects/FSharp.Control.AsyncSeq/pull/116}{https://github.com/fsprojects/FSharp.Control.AsyncSeq/pull/116}


\end{document}