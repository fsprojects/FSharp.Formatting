\documentclass{article}

\usepackage{xcolor}
\usepackage{graphicx}
\usepackage{fancyvrb}
\usepackage{listings}
\usepackage[T1]{fontenc}
\usepackage{hyperref}
\usepackage{amsmath}

\definecolor{officegreen}{rgb}{0, 0.5, 0}
\definecolor{navy}{rgb}{0, 0, 0.5}
\definecolor{linecolor}{rgb}{0.5, 0.6875, 0.6875}
\definecolor{outputcolor}{rgb}{0.375, 0.375, 0.375}

\newcommand{\id}[1]{\textcolor{black}{#1}}
\newcommand{\com}[1]{\textcolor{officegreen}{#1}}
\newcommand{\inact}[1]{\textcolor{gray}{#1}}
\newcommand{\kwd}[1]{\textcolor{navy}{#1}}
\newcommand{\num}[1]{\textcolor{officegreen}{#1}}
\newcommand{\ops}[1]{\textcolor{purple}{#1}}
\newcommand{\prep}[1]{\textcolor{purple}{#1}}
\newcommand{\str}[1]{\textcolor{olive}{#1}}
\newcommand{\lines}[1]{\textcolor{linecolor}{#1}}
\newcommand{\fsi}[1]{\textcolor{outputcolor}{#1}}
\newcommand{\omi}[1]{\textcolor{gray}{#1}}

% Overriding color and style of line numbers
\renewcommand{\theFancyVerbLine}{
\lines{\small \arabic{FancyVerbLine}:}}

\lstset{%
  backgroundcolor=\color{gray!15},
  basicstyle=\ttfamily,
  breaklines=true,
  columns=fullflexible
}

\title{{page-title}}
\date{}

\begin{document}

\maketitle



\href{https://mybinder.org/v2/gh/fsprojects/FSharp.Formatting/gh-pages?filepath=literate.ipynb}{\begin{figure}[htbp]\centering
\includegraphics[width=1.0\textwidth]{https://mybinder.org/badge\_logo.svg}
\caption{Binder}
\end{figure}
}
\section*{Creating Content}



The \href{commandline.html}{"fsdocs" tool} allows documentation for a site to be built
from content in a \texttt{docs} directory. The expected structure for a \texttt{docs} directory is
\begin{lstlisting}
docs/**/*.md                  -- markdown with embedded code, converted to html and optionally tex/ipynb
docs/**/*.fsx                 -- fsx scripts converted to html and optionally tex/ipynb
docs/**/*                     -- other content, copied over
docs/**/_template.html        -- optional template, specifies the HTML template for this directory and its contents
docs/**/_template.tex         -- optionally indicates Latex files should be generated
docs/**/_template.ipynb       -- optionally indicates F# ipynb files should be generated
docs/**/_template.fsx         -- optionally indicates F# fsx files should be generated (even from markdown)
docs/reference/_template.html -- optionally specifies the default template for reference docs

\end{lstlisting}


Processing is by these two commands:
\begin{Verbatim}[commandchars=\\\{\}]
\id{dotnet} \id{fsdocs} \id{build}
\id{dotnet} \id{fsdocs} \id{watch}

\end{Verbatim}



The output goes in \texttt{output/} by default.  Processing is recursive, making this a form of static site generation.
\subsection*{Literate Scripts and Markdown}



The input directory may contain \href{literate.html}{literate scripts and markdown}.
\subsection*{Other Content}



Content that is not \texttt{*.fsx} or \texttt{*.md} is copied across.
\subsection*{Default Styling Content}



By default additional content such as \texttt{fsdocs-search.js}, \texttt{fsdocs-tips.js} and \texttt{fsdocs-styles.css} are included in the
the \texttt{content} directory of the output.  This can be suppressed with \texttt{--nodefaultcontent} or by having your own
copy of this content in your \texttt{content} directory.
\subsection*{Ignored Content}



Any file or directory beginning with \texttt{.} is ignored.
\subsection*{HTML Templates}



Template files are named \texttt{\_template.html} and should contain \texttt{\{\{fsdocs-content\}\}},  \texttt{\{\{fsdocs-tooltips\}\}}
and other placeholders.
If a file \texttt{\_template.html} exists then is used as the template for HTML generation for that directory and all sub-content.
Otherwise the default template is used.


The following substitutions determine the primary (non-styling) content of your site.
For example \texttt{\{\{fsdocs-content\}\}} is replaced with the generated content.


See \href{styling.html}{Styling} for information about template parameters and styling beyond the default template.
\begin{tabular}{|l|l|}\hline
\textbf{Substitution name} & \textbf{Generated content}\\ \hline\hline
\texttt{root} & \texttt{<RepositoryUrl>} else \texttt{/}\\ \hline
\texttt{fsdocs-content} & Main page content\\ \hline
\texttt{fsdocs-list-of-namespaces} & HTML \texttt{<li>} list of namespaces with links\\ \hline
\texttt{fsdocs-list-of-documents} & HTML \texttt{<li>} list of documents with  titles and links\\ \hline
\texttt{fsdocs-page-title} & First h1 heading in literate file. Generated for API docs\\ \hline
\texttt{fsdocs-source} & Original script source\\ \hline
\texttt{fsdocs-tooltips} & Generated hidden div elements for tooltips\\ \hline
\end{tabular}



The following substitutions are extracted from your project files and may or may not be used by the default
template:
\begin{tabular}{|l|l|}\hline
\textbf{Substitution name} & \textbf{Value}\\ \hline\hline
\texttt{fsdocs-copyright} & \texttt{<Copyright>}\\ \hline
\texttt{fsdocs-package-project-url} & \texttt{<PackageProjectUrl>}\\ \hline
\texttt{fsdocs-package-license-expression} & \texttt{<PackageLicenseExpression>}\\ \hline
\texttt{fsdocs-package-tags} & \texttt{<PackageTags>}\\ \hline
\texttt{fsdocs-package-version} & \texttt{<Version>}\\ \hline
\end{tabular}

\subsection*{Generating LaTeX output}



For Latex, the the \texttt{\_template.tex} file is either empty of contains \texttt{\{content\}} as the key where the body
of the document is placed.


To generate .tex output for each script and markdown file, add a \texttt{\_template.tex}.
It may contain \texttt{\{\{fsdocs-content\}\}}.
\subsection*{Generating iPython Notebook output}



To generate .ipynb output for each script and markdown file, add a \texttt{\_template.ipynb}, usually empty.


To add a \texttt{mybinder} badge to your generated notebook, ensure you have a \texttt{Dockerfile} and \texttt{NuGet.config}
in your \texttt{docs} directory and use text like this:
\begin{Verbatim}[commandchars=\\\{\}]
{[}\ops{!}{[}\id{Binder}{]}{(}\id{https}{:}\com{//mybinder.org/badge\_logo.svg)](https://mybinder.org/v2/gh/fsprojects/FSharp.Formatting/gh-pages?filepath=literate.ipynb)}

\end{Verbatim}

\subsection*{Generating Script outputs}



To generate .fsx output for each script and markdown file, add a \texttt{\_template.fsx}, usually empty.
It may contain \texttt{\{\{fsdocs-content\}\}}.


\end{document}