\documentclass{article}

\usepackage{xcolor}
\usepackage{graphicx}
\usepackage{fancyvrb}
\usepackage{listings}
\usepackage[T1]{fontenc}
\usepackage{hyperref}
\usepackage{amsmath}

\definecolor{officegreen}{rgb}{0, 0.5, 0}
\definecolor{navy}{rgb}{0, 0, 0.5}
\definecolor{linecolor}{rgb}{0.5, 0.6875, 0.6875}
\definecolor{outputcolor}{rgb}{0.375, 0.375, 0.375}

\newcommand{\id}[1]{\textcolor{black}{#1}}
\newcommand{\com}[1]{\textcolor{officegreen}{#1}}
\newcommand{\inact}[1]{\textcolor{gray}{#1}}
\newcommand{\kwd}[1]{\textcolor{navy}{#1}}
\newcommand{\num}[1]{\textcolor{officegreen}{#1}}
\newcommand{\ops}[1]{\textcolor{purple}{#1}}
\newcommand{\prep}[1]{\textcolor{purple}{#1}}
\newcommand{\str}[1]{\textcolor{olive}{#1}}
\newcommand{\lines}[1]{\textcolor{linecolor}{#1}}
\newcommand{\fsi}[1]{\textcolor{outputcolor}{#1}}
\newcommand{\omi}[1]{\textcolor{gray}{#1}}

% Overriding color and style of line numbers
\renewcommand{\theFancyVerbLine}{
\lines{\small \arabic{FancyVerbLine}:}}

\lstset{%
  backgroundcolor=\color{gray!15},
  basicstyle=\ttfamily,
  breaklines=true,
  columns=fullflexible
}

\title{{page-title}}
\date{}

\begin{document}

\maketitle


\section*{Example: Using the Markdown Extensions for LaTeX}



<script id="MathJax-script" async src="https://cdn.jsdelivr.net/npm/mathjax@3.0.1/es5/tex-mml-chtml.js"></script>


To use LaTex extension, you need to add javascript
link to \href{http://www.mathjax.org/}{MathJax} in
your template or inside a \texttt{\_head.html} file.
\begin{lstlisting}
<script id="MathJax-script" async src="https://cdn.jsdelivr.net/npm/mathjax@3.0.1/es5/tex-mml-chtml.js"></script>

\end{lstlisting}


To use inline LaTex, eclose LaTex code with \texttt{\$}:
$k_{n+1} = n^2 + k_n^2 - k_{n-1}$. Alternatively,
you can also use \texttt{\$\$}.


To use block LaTex, start a new paragraph, with
the first line marked as \texttt{\$\$\$} (no close \texttt{\$\$\$}):


\begin{equation}
A_{m,n} =
 \begin{pmatrix}
  a_{1,1} & a_{1,2} & \cdots & a_{1,n} \\
  a_{2,1} & a_{2,2} & \cdots & a_{2,n} \\
  \vdots  & \vdots  & \ddots & \vdots  \\
  a_{m,1} & a_{m,2} & \cdots & a_{m,n}
 \end{pmatrix}
\end{equation}




Use LaTex escape rule:
\begin{itemize}
\item Escape \$ in inline mode: $\$$, $\$var$

\item Other escapes: $\& \% \$ \# \_ \{ \}$

\item Using < or >: $x > 1$, $y < 1$, $x >= 1$,
$y <= 1$, $x = 1$

\item $<p>something</p>$

\end{itemize}



\end{document}