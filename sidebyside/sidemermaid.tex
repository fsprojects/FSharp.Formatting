\documentclass{article}

\usepackage{xcolor}
\usepackage{graphicx}
\usepackage{fancyvrb}
\usepackage{listings}
\usepackage[T1]{fontenc}
\usepackage{hyperref}
\usepackage{amsmath}

\definecolor{officegreen}{rgb}{0, 0.5, 0}
\definecolor{navy}{rgb}{0, 0, 0.5}
\definecolor{linecolor}{rgb}{0.5, 0.6875, 0.6875}
\definecolor{outputcolor}{rgb}{0.375, 0.375, 0.375}

\newcommand{\id}[1]{\textcolor{black}{#1}}
\newcommand{\com}[1]{\textcolor{officegreen}{#1}}
\newcommand{\inact}[1]{\textcolor{gray}{#1}}
\newcommand{\kwd}[1]{\textcolor{navy}{#1}}
\newcommand{\num}[1]{\textcolor{officegreen}{#1}}
\newcommand{\ops}[1]{\textcolor{purple}{#1}}
\newcommand{\prep}[1]{\textcolor{purple}{#1}}
\newcommand{\str}[1]{\textcolor{olive}{#1}}
\newcommand{\lines}[1]{\textcolor{linecolor}{#1}}
\newcommand{\fsi}[1]{\textcolor{outputcolor}{#1}}
\newcommand{\omi}[1]{\textcolor{gray}{#1}}

% Overriding color and style of line numbers
\renewcommand{\theFancyVerbLine}{
\lines{\small \arabic{FancyVerbLine}:}}

\lstset{%
  backgroundcolor=\color{gray!15},
  basicstyle=\ttfamily,
  breaklines=true,
  columns=fullflexible
}

\title{{page-title}}
\date{}

\begin{document}

\maketitle


\section*{Example: Mermaid Diagrams}



\href{https://mermaid.js.org/}{Mermaid} is a JavaScript-based diagramming and charting tool that renders Markdown-inspired text definitions into diagrams.
\subsection*{Setup}



Add the Mermaid JavaScript library to your site by creating or editing a \texttt{\_head.html} file in your \texttt{docs} folder:
\begin{lstlisting}
<script type="module">
  import mermaid from 'https://cdn.jsdelivr.net/npm/mermaid@11/dist/mermaid.esm.min.mjs';
</script>

\end{lstlisting}
\subsection*{Usage}



To embed a Mermaid diagram, wrap your Mermaid syntax in a \texttt{<div>} element with the \texttt{mermaid} CSS class:
\begin{lstlisting}
<div class="mermaid">
graph LR
    A[Input docs] --> B[fsdocs build]
    B --> C[HTML output]
    B --> D[API reference]
</div>

\end{lstlisting}


This renders as:
<div class="mermaid">
graph LR
    A[Input docs] --> B[fsdocs build]
    B --> C[HTML output]
    B --> D[API reference]
</div>
\subsection*{More Examples}



Sequence diagram:
<div class="mermaid">
sequenceDiagram
    participant User
    participant fsdocs
    participant Browser
    User->>fsdocs: dotnet fsdocs watch
    fsdocs-->>Browser: Serve docs
    User->>fsdocs: Edit .md or .fsx
    fsdocs-->>Browser: Reload page
</div>


Class diagram:
<div class="mermaid">
classDiagram
    class ApiDocComment {
        +Summary: string
        +Remarks: string option
        +Parameters: ApiDocSection list
    }
    class ApiDocMember {
        +Name: string
        +Comment: ApiDocComment
    }
    ApiDocMember --> ApiDocComment
</div>
\subsection*{Tips}

\begin{itemize}
\item You can also use \texttt{<div class="mermaid text-center">} to centre the diagram on the page.

\item To customise the Mermaid theme, pass options to \texttt{mermaid.initialize()} before the \texttt{import} call.

\item See the \href{https://mermaid.js.org/intro/}{Mermaid documentation} for the full list of supported diagram types.

\end{itemize}



\end{document}