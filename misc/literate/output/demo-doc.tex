\documentclass{article}

% Defining colors by names
\usepackage{xcolor}
% Verbatim enviroment
\usepackage{fancyvrb}
% Verbatim enviroment for unformatted source code
\usepackage{listings}
% Better font for backslash
\usepackage[T1]{fontenc}
\usepackage{hyperref}
% Providing more features than usual tabular
\usepackage{longtable}

% Identifiers in color code #000000
\newcommand{\id}[1]{\textcolor{black}{#1}}

% Comments in green
\definecolor{officegreen}{rgb}{0, 0.5, 0}
\newcommand{\com}[1]{\textcolor{officegreen}{#1}}

% Inactive elements in color code #808080
\newcommand{\inact}[1]{\textcolor{gray}{#1}}

% Keywords in color code #000080
\definecolor{navy}{rgb}{0, 0, 0.5}
\newcommand{\kwd}[1]{\textcolor{navy}{#1}}

% Numbers in color code #008000
\newcommand{\num}[1]{\textcolor{officegreen}{#1}}

% Operators in color code #800080
\newcommand{\ops}[1]{\textcolor{purple}{#1}}

% Preprocessors in color code #800080
\newcommand{\prep}[1]{\textcolor{purple}{#1}}

% Strings in color code #808000
\newcommand{\str}[1]{\textcolor{olive}{#1}}

% Lines in color code #80b0b0
% Define relative color to work correctly with \newcommand
\definecolor{linecolor}{rgb}{0.5, 0.6875, 0.6875}
\newcommand{\lines}[1]{\textcolor{linecolor}{#1}}

% fsi output in color code #606060
\definecolor{outputcolor}{rgb}{0.375, 0.375, 0.375}
\newcommand{\fsi}[1]{\textcolor{outputcolor}{#1}}

% Omitted parts in color code #808080
\newcommand{\omi}[1]{\textcolor{gray}{#1}}

% Overriding color and style of line numbers
\renewcommand{\theFancyVerbLine}{
\lines{\small \arabic{FancyVerbLine}:}}

\lstset{%
  backgroundcolor=\color{gray!15},
  basicstyle=\ttfamily,
  breaklines=true,
  columns=fullflexible
}

\title{Literate sample
}
\date{}

\begin{document}

\maketitle

\section*{Literate sample}



This file demonstrates how to write Markdown document with 
embedded F\# snippets that can be transformed into nice HTML 
using the \texttt{literate.fsx} script from the \href{http://tpetricek.github.com/FSharp.Formatting}{F\# Formatting
package}.


In this case, the document itself is a valid Markdown and 
you can use standard Markdown features to format the text:
\begin{itemize}
\item Here is an example of unordered list and...

\item Text formatting including \textbf{bold} and \emph{emphasis}

\end{itemize}



For more information, see the \href{http://daringfireball.net/projects/markdown}{Markdown} reference.
\subsection*{Writing F\# code}



In standard Markdown, you can include code snippets by 
writing a block indented by four spaces and the code 
snippet will be turned into a \texttt{<pre>} element. If you do 
the same using Literate F\# tool, the code is turned into
a nicely formatted F\# snippet:
\begin{Verbatim}[commandchars=\\\{\}, numbers=left]
\com{/// The Hello World of functional languages!}
\kwd{let} \kwd{rec} {factorial} \id{x} \ops{=} 
  \kwd{if} \id{x} \ops{=} \num{0} \kwd{then} \num{1} 
  \kwd{else} \id{x} \ops{*} ({factorial} (\id{x} \ops{-} \num{1}))

\kwd{let} \id{f10} \ops{=} {factorial} \num{10}

\end{Verbatim}

\subsection*{Hiding code}



If you want to include some code in the source code, 
but omit it from the output, you can use the \texttt{hide} 
command. You can also use \texttt{module=...} to specify that 
the snippet should be placed in a separate module 
(e.g. to avoid duplicate definitions).


The value will be deffined in the F\# code that is 
processed and so you can use it from other (visible) 
code and get correct tool tips:
\begin{Verbatim}[commandchars=\\\{\}, numbers=left]
\kwd{let} \id{answer} \ops{=} {Hidden}\ops{.}\id{answer}

\end{Verbatim}

\subsection*{Including other snippets}



When writing literate programs as Markdown documents, 
you can also include snippets in other languages. 
These will not be colorized and processed as F\# 
code samples:
\begin{lstlisting}
Console.WriteLine("Hello world!");

\end{lstlisting}


This snippet is turned into a \texttt{pre} element with the
\texttt{lang} attribute set to \texttt{csharp}.




\end{document}
