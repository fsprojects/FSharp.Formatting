\documentclass{article}

\usepackage{xcolor}
\usepackage{graphicx}
\usepackage{fancyvrb}
\usepackage{listings}
\usepackage[T1]{fontenc}
\usepackage{hyperref}
\usepackage{amsmath}

\definecolor{officegreen}{rgb}{0, 0.5, 0}
\definecolor{navy}{rgb}{0, 0, 0.5}
\definecolor{linecolor}{rgb}{0.5, 0.6875, 0.6875}
\definecolor{outputcolor}{rgb}{0.375, 0.375, 0.375}

\newcommand{\id}[1]{\textcolor{black}{#1}}
\newcommand{\com}[1]{\textcolor{officegreen}{#1}}
\newcommand{\inact}[1]{\textcolor{gray}{#1}}
\newcommand{\kwd}[1]{\textcolor{navy}{#1}}
\newcommand{\num}[1]{\textcolor{officegreen}{#1}}
\newcommand{\ops}[1]{\textcolor{purple}{#1}}
\newcommand{\prep}[1]{\textcolor{purple}{#1}}
\newcommand{\str}[1]{\textcolor{olive}{#1}}
\newcommand{\lines}[1]{\textcolor{linecolor}{#1}}
\newcommand{\fsi}[1]{\textcolor{outputcolor}{#1}}
\newcommand{\omi}[1]{\textcolor{gray}{#1}}

% Overriding color and style of line numbers
\renewcommand{\theFancyVerbLine}{
\lines{\small \arabic{FancyVerbLine}:}}

\lstset{%
  backgroundcolor=\color{gray!15},
  basicstyle=\ttfamily,
  breaklines=true,
  columns=fullflexible
}

\title{{page-title}}
\date{}

\begin{document}

\maketitle

\section*{Command line tool}



To use F\# Formatting tools via the command line, you can use the \texttt{fsdocs} dotnet tool.
\begin{Verbatim}[commandchars=\\\{\}]
\id{dotnet} \id{tool} \id{install} \id{FSharp}{.}\id{Formatting}{.}\id{CommandTool}
\id{dotnet} \id{fsdocs} {[}\id{command}{]} {[}\id{options}{]}

\end{Verbatim}

\subsection*{The build command}



This command processes a \texttt{docs} directory and generates API docs for projects in the solution according to the
rules of \href{apidocs.html}{API doc generation}
\begin{lstlisting}
fsdocs build

\end{lstlisting}


The input accepted is described in \href{content.html}{content}.


The command line options accepted are:
\begin{tabular}{|l|c|}\hline
\textbf{Option} & \textbf{Description}\\ \hline\hline
--input & Input directory of documentation content (default: \texttt{docs})\\ \hline
--projects & Project files to build API docs for outputs, defaults to all packable projects.\\ \hline
--output & Output Directory (default 'output' for 'build' and 'tmp/watch' for 'watch'.\\ \hline
--noapidocs & (Default: false) Disable generation of API docs.\\ \hline
--eval & (Default: false) Evaluate F\# fragments in scripts.\\ \hline
\end{tabular}



| --saveimages        |        (Default: none) Save images referenced in docs (some|none|all). If 'some' then image links in formatted results are saved for latex and ipynb output docs. |
| --nolinenumbers       |      Don't add line numbers, default is to add line numbers. |
| --parameters            |    Additional substitution parameters for templates. |
| --nonpublic           |      (Default: false) The tool will also generate documentation for non-public members |
| --nodefaultcontent      |    Do not copy default content styles, javascript or use default templates. |
| --clean                 |    (Default: false) Clean the output directory. |
| --help                  |    Display this help screen. |
| --version               |    Display version information. |


The following command line options are also accepted but it is instead recommended you use
settings in your .fsproj project files:
\begin{tabular}{|l|c|}\hline
\textbf{Option} & \textbf{Description}\\ \hline\hline
--sourcefolder & Source folder at time of component build (defaults to value of \texttt{<FsDocsSourceFolder>} from project file, else current directory)\\ \hline
--sourcerepo & Source repository for github links (defaults to value of \texttt{<FsDocsSourceRepository>} from project file, else \texttt{<RepositoryUrl>/tree/<RepositoryBranch>} for Git repositories)\\ \hline
--mdcomments & Assume /// comments in F\# code are markdown style (defaults to value of \texttt{<UsesMarkdownComments>} from project file)\\ \hline
\end{tabular}

\subsection*{The watch command}



This command does the same as \texttt{fsdocs build} but in "watch" mode, waiting for changes. Only the files in the input
directory (e.g. \texttt{docs}) are watched.
\begin{lstlisting}
fsdocs watch

\end{lstlisting}


Restarting may be necesssary on changes to project files. The same parameters are accepted, plus these:
\begin{tabular}{|l|c|}\hline
\textbf{Option} & \textbf{Description}\\ \hline\hline
\texttt{--noserver} & (Default: false) Do not serve content when watching.\\ \hline
\texttt{--nolaunch} & (Default: false) Do not launch a browser window.\\ \hline
\texttt{--open} & (Default: ) URL extension to launch \href{http://localhost:<port>/\%s.}{http://localhost:<port>/\%s.}\\ \hline
\texttt{--port} & (Default: 8901) Port to serve content for \href{http://localhost}{http://localhost} serving.\\ \hline
\end{tabular}

\subsection*{Searchable docs}



When using the command-line tool a Lunr search index is automatically generated in \texttt{index.json}.


A search box is included in the default template.  To add a search box
to your own \texttt{\_template.html}, include \texttt{fsdocs-search.js}, which is added to the \texttt{content}
by default.
\begin{Verbatim}[commandchars=\\\{\}]
\ops{..}{.}
{<}\id{div} \id{id}\ops{=}\str{"header"}{>}
  {<}\id{div} \kwd{class}\ops{=}\str{"searchbox"}{>}
    {<}\id{label} \kwd{for}\ops{=}\str{"search-by"}{>}
      {<}\id{i} \kwd{class}\ops{=}\str{"fas fa-search"}{>}\ops{</}\id{i}{>}
    \ops{</}\id{label}{>}
    {<}\id{input} \id{data}\ops{-}\id{search}\ops{-}\id{input}\ops{=}\str{""} \id{id}\ops{=}\str{"search-by"} \kwd{type}\ops{=}\str{"search"} \id{placeholder}\ops{=}\str{"Search..."} \ops{/>}
    {<}\id{span} \id{data}\ops{-}\id{search}\ops{-}\id{clear}\ops{=}\str{""}{>}
      {<}\id{i} \kwd{class}\ops{=}\str{"fas fa-times"}{>}\ops{</}\id{i}{>}
    \ops{</}\id{span}{>}
  \ops{</}\id{div}{>}
\ops{</}\id{div}{>}
\ops{..}{.}

\end{Verbatim}



\end{document}