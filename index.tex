\documentclass{article}

\usepackage{xcolor}
\usepackage{graphicx}
\usepackage{fancyvrb}
\usepackage{listings}
\usepackage[T1]{fontenc}
\usepackage{hyperref}
\usepackage{amsmath}

\definecolor{officegreen}{rgb}{0, 0.5, 0}
\definecolor{navy}{rgb}{0, 0, 0.5}
\definecolor{linecolor}{rgb}{0.5, 0.6875, 0.6875}
\definecolor{outputcolor}{rgb}{0.375, 0.375, 0.375}

\newcommand{\id}[1]{\textcolor{black}{#1}}
\newcommand{\com}[1]{\textcolor{officegreen}{#1}}
\newcommand{\inact}[1]{\textcolor{gray}{#1}}
\newcommand{\kwd}[1]{\textcolor{navy}{#1}}
\newcommand{\num}[1]{\textcolor{officegreen}{#1}}
\newcommand{\ops}[1]{\textcolor{purple}{#1}}
\newcommand{\prep}[1]{\textcolor{purple}{#1}}
\newcommand{\str}[1]{\textcolor{olive}{#1}}
\newcommand{\lines}[1]{\textcolor{linecolor}{#1}}
\newcommand{\fsi}[1]{\textcolor{outputcolor}{#1}}
\newcommand{\omi}[1]{\textcolor{gray}{#1}}

% Overriding color and style of line numbers
\renewcommand{\theFancyVerbLine}{
\lines{\small \arabic{FancyVerbLine}:}}

\lstset{%
  backgroundcolor=\color{gray!15},
  basicstyle=\ttfamily,
  breaklines=true,
  columns=fullflexible
}

\title{{page-title}}
\date{}

\begin{document}

\maketitle



\href{https://mybinder.org/v2/gh/fsprojects/FSharp.Formatting/gh-pages?filepath=index.ipynb}{\begin{figure}[htbp]\centering
\includegraphics[width=1.0\textwidth]{https://mybinder.org/badge\_logo.svg}
\caption{Binder}
\end{figure}
}
\section*{F\# Formatting: Documentation Tools for F\# Code}



FSharp.Formatting is a set of libraries and tools for processing F\# script files, markdown and for
generating API documentation. F\# Formatting package is used by the this project and many other repositories.


To use the tool, install and use the \href{commandline.html}{fsdocs} tool in a typical F\# project with
F\# project files plus markdown and script content in the \texttt{docs} directory:
\begin{Verbatim}[commandchars=\\\{\}]
\id{dotnet} \id{tool} \id{install} \id{FSharp}{.}\id{Formatting}{.}\id{CommandTool}
\id{dotnet} \id{fsdocs} \id{build} 
\id{dotnet} \id{fsdocs} \id{watch}

\end{Verbatim}



To use the tool, explore the following topics:
\begin{itemize}
\item 

\href{content.html}{Authoring Content} - explains the content expected in the \texttt{docs} directory for the \texttt{fsdocs} tool.

\item 

\href{commandline.html}{Using the Command line tool} - explains how to use the \texttt{fsdocs} tool.

\item 

\href{apidocs.html}{Generating API documentation} - how to generate HTML documentation
from "XML comments" in your .NET libraries. The tool handles comments written in
Markdown too.

\item 

\href{styling.html}{Styling} - explains some options for styling the output of \texttt{fsdocs}.

\item 

\href{literate.html}{Using literate programming} - explains how to generate documentation
for your projects or to write nicely formatted F\# blog posts.

\item 

\href{evaluation.html}{Embedding F\# outputs in literate programming} - provides more details on literate programming and
explains how to embed results of a literate script file in the generated output. This way,
you can easily format the results of running your code!

\end{itemize}

\subsection*{Using FSharp.Formatting as a library}



F\# Formatting is also \href{https://nuget.org/packages/FSharp.Formatting}{available on NuGet} as a set of libraries.
\begin{itemize}
\item 

\href{markdown.html}{Markdown parser} - explains the F\# Markdown
processor that is available in this library with some basic examples of
document processing.

\item 

\href{codeformat.html}{F\# code formatting} - more details about the F\# code
formatter and how to use it to obtain information about F\# source files.

\end{itemize}

\subsection*{More information}



The documentation for this library is generated automatically using the tools
built here. If you spot a typo, please submit a pull request! The source Markdown and F\# script files are
available in the \href{https://github.com/fsprojects/FSharp.Formatting/tree/master/docs}{docs folder on GitHub}.


The project is hosted on \href{https://github.com/fsprojects/FSharp.Formatting}{GitHub} where you can
\href{https://github.com/fsprojects/FSharp.Formatting/issues}{report issues}, fork the project and submit pull requests.
See the  \href{https://github.com/fsprojects/FSharp.Formatting/blob/master/LICENSE.md}{License file} in the GitHub repository.


\end{document}